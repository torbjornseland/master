\documentclass[norsk]{beamer}

\usepackage[utf8]{inputenc}
\usepackage{babel}
\usepackage{lipsum}
\usepackage{amsmath}
%\usepackage{mathtools}

\usetheme{Warsaw}
\title[Løse likningssett med matriseregning]{Matriseregning\\ Introduksjon i matriseregning for Ent3relever}
\author{Torbjørn Seland}

\newcommand\Fontvi{\fontsize{20}{7.2}\selectfont}
\newcommand\Fontvitwo{\fontsize{10}{7.2}\selectfont}
\newcommand\Fontvithree{\fontsize{15}{7.2}\selectfont}

\begin{document}

\begin{frame}
\titlepage
\end{frame}

\begin{frame}{Info om opplegget til studenter}
\begin{itemize}
	\item Dette opplegget ble laget for en 1.klasse. Ideen var å vise hvordan
		mer kompliserte likningssett kan løses på en mer effektiv måte
	\item Jeg brukte tavlen aktivt da jeg forklarte opplegget og denne
		powerpointen er kun til støtte. Dette krever selvfølgelig at man selv
		har god kontroll på temaet.
	\item Gå gjennom eksempeloppgaven grundig og forklar hvert steg. Her er det
		viktig at elevene skjønner teknikken og ser sammenhengen.
	\item Jeg har laget tre oppgaver med fasit i slutten av presentasjonen.
		Disse er laget slik at de skal kunne regnes på en fin måte(om man gjør
		det riktig). I den siste oppgaven skal elevene også gjøre om fra et
		likningssystem til et matriseoppsett.
\end{itemize}
\end{frame}

\Fontvi
\begin{frame}{Likningssett}
\begin{align*}
I: 2x + y =& 5\\
II: x + 2y =& 4
\end{align*}

\Fontvitwo
\begin{itemize}
	\item Hvordan ville dere løst dette likningssettet?
\end{itemize}
\end{frame}
\begin{frame}{Fra likningssett til matriseoppsett}
\Fontvi
\centering
\begin{align*}
\begin{matrix}
I:& 2x& + y& =& 5\\
II:& x& + 2y& =& 4
\end{matrix}
\end{align*}
\\
\Downarrow\\
\\
\begin{align*}
\begin{bmatrix}
	2 & 1 \\
	1 & 2
\end{bmatrix}
\begin{bmatrix}
	x \\
	y
\end{bmatrix}
=
\begin{bmatrix}
	5 \\
	4
\end{bmatrix}
\end{align*}
\end{frame}
\begin{frame}
\Fontvi
\begin{align*}
\begin{bmatrix}
	1 & 0 \\
	0 & 1
\end{bmatrix}
\begin{bmatrix}
	x \\
	y
\end{bmatrix}
=
\begin{bmatrix}
	a \\
	b
\end{bmatrix}
\end{align*}
\Fontvitwo
\begin{itemize}
	\item Vi ønsker å ende opp med en slik matrise. Hvorfor?
	\item Dette kan vi klare ved å trekke fra og legge til radene på
		hverandre(vannrett).
	\item Denne matrisen kalles en identitetsmatrise/enhetsmatrise
\end{itemize}
\end{frame}
\begin{frame}
\Fontvithree
Oppgaver\\
a)
\begin{align*}
\begin{bmatrix}
	1 & 3 \\
	3 & 1
\end{bmatrix}
\begin{bmatrix}
	x \\
	y
\end{bmatrix}
=
\begin{bmatrix}
	6 \\
	10
\end{bmatrix}\\
\end{align*}
b)
\begin{align*}
\begin{bmatrix}
	2 & 2 \\
	3 & 1
\end{bmatrix}
\begin{bmatrix}
	x \\
	y
\end{bmatrix}
=
\begin{bmatrix}
	12 \\
	10
\end{bmatrix}\\
\end{align*}
c)
\begin{align*}
	x + 2y + z &=9\\
      + 2y + z &=7\\
	x + y  + z &=6
\end{align*}
\end{frame}
\begin{frame}{Fasit oppgaver}
\begin{itemize}
	\item a) x= 3, y=1
	\item b) x= 2, y=4
	\item c) x= 2, y=3, z=1
\end{itemize}
\end{frame}

\end{document}
