%%
%% Automatically generated ptex2tex (extended LaTeX) file
%% from Doconce source
%% http://code.google.com/p/doconce/
%%

\documentclass{article}

\usepackage{relsize,epsfig,makeidx,amsmath,amsfonts}
\usepackage[latin1]{inputenc}
\usepackage{ptex2tex}


\usepackage[%
colorlinks=true,
linkcolor=blue,
citecolor=black,
filecolor=blue,
urlcolor=blue]{hyperref}
%\hyperbaseurl{}   % hyperlinks are relative to this root

\newcommand{\inlinecomment}[2]{  ({\bf #1}: \emph{#2})  }
%\newcommand{\inlinecomment}[2]{}  % turn off inline comments

\makeindex

\begin{document}




% ----------------- Title -------------------------

\begin{center}
{\LARGE\bf Project inf5631}
\end{center}




% ----------------- Author(s) -------------------------

\begin{center}
{\bf Torbjørn Seland${}^{}$} \\ [0mm]
\end{center}

\begin{center}
% List of all institutions:
\end{center}
% ----------------- End of author(s) -------------------------



% ----------------- Date -------------------------


\begin{center}
May 27, 2014
\end{center}


TOC: on


\section{The model: Reaction diffusion equation}

The strength of calculations by computers has increased enormously the last
decades. This is something that has given us new
tools in how to describe everything around us. This has also affected the way to
look at biological and medical issues. This
paper will focus on the numerically way to solve it and how we can use
mathematical tools to describe and analyse aa biological model. I will look at 
the reaction diffusion equation (RD):
\begin{align} \label{eq:reaction_diffusion}
u_t =& \nabla \cdot \alpha(u)\nabla u + f(u)\\
u(x,0) =& I(x)\\
u(0,t)=& 0\\
u(L,t)=& 0
\end{align}
This describe the concentration $u$ in
one spatial dimension.To get a greater understanding, we can split up the model. The right side of the equation consist of two parts. The
first part describe the spatial flow while the second describe the reaction
term.


\section{Reaction term}

Here I want to investigate the reaction term. This describe the concentration in
a certain point.
\begin{equation}
u_t = f(u)
\end{equation}
To see the properties of this equation, we can  discretize the equation and solve it numerically for
different values of $f(u)$.

\begin{align*}
u_t =& f(u)\\
\frac{u^n_i-u^{n-1}_i}{\Delta t}=& f(u^n_i)\\
u^n_i-\Delta t f(u^n_i) =& u^{n-1}_i
\end{align*}
This gives us a linear equation $Au = b$ that easy can be solved.

\subsection{Fisher-Kolmogoroff equation}

When this part consist as the reaction term in our RD it is called Fisher-Kolmogoroff. This
is often used in biology.Here $r$ describe the linear reproduction and
$m$ the carrying capacity of this environment. The solution
will then be.
\begin{equation}
f(u) = ru(1-\frac{u}{m})
\end{equation}


\begin{align*}
u^n_i-\Delta t ru^n_i(1-\frac{u^n_i}{m}) =& u^{n-1}_i\\
u^n_i(1-\Delta t r(1-\frac{u^n_i}{m})) =& u^{n-1}_i
\end{align*}

Since this equation gives us a nonlinear equation, we can use Picard to solve
it. The idea with Picard is to replace one of the $u$ in the nonlinear part of the equation with $u\_$. $u\_$ will in the first
iteration be sat to the value from the previous step $u\_1$. 

\begin{equation*}
	u\_ = u^{n-1}
\end{equation*}

We can check the correct value $u$ against our pre produced $u\_$ each round. If the difference between
them are less than what we demand, it continue. If not, the new $u\_$ will be a combination of $u\_$ and $u$. How we weight the combination is called relaxation 

\begin{equation*}
u\_ = \gamma u + (1-\gamma)u\_, 0<= \gamma <=1
\end{equation*}

Our new equation will then be

\begin{align*}
	u^n_i(1- \Delta t r(1-\frac{u\__i^n}{m}))=& u^{n-1}_i\\
\end{align*}

Since we are using an approximation to $u$, we need to refine $u\_$ until it
fulfil our expectations.

\paragraph{Fisher's equation.}
If we set $r=1$ and $m=1$, we will get a equation called Fisher's equation. This is the
original equation for the FK equation.The reaction term in Fisher will look like this

% #MOVIE: [movies/time_der_r_1_0_m_1_0.webm, height=300,width=375] r= 1, m= 1

\paragraph{Carrying capacity.}
First it can be interesting to look at the changes in the carrying capacity of
the function.

% #MOVIE: [paramovies/time.webm, height=300,width=375]time

Here we can see that they all goes towards their carrying capacity and stabilizes here.
I have set $r=1$ in all of this simulations. The displacement from intial conditions to
final conditions is quite similar in all of the cases.

\paragraph{Reproduction rate.}
We can do a similar thing with the reproduction $r$.

% #MOVIE: [paramovies/time_r.webm, height=300,width=375]time_r

This controls the speed of the movement. A larger reproduction will get in faster
to the steady state, that $m$ controls.The carrying capacity is sat to $m=0.8$.


\section{Spatial diffusion}

The first part of the RD equation describe the spatial flow and how the values in $u$ are
affected by their neighbours.  
\begin{equation}
	u_t = \nabla \cdot \alpha(u)\nabla u
\end{equation}
This part consist of the function $\alpha(u)$. This
affects the diffusion of the model. 
To solve this numerically, we need to discretize the equation.
\begin{equation*}
	\left[D_t^-u=D_x(\alpha(u) D_x u)\right]
\end{equation*}
I use Backward Euler for the time discrete and Crank Nicolson for the spatial
discrete. 
\begin{align*}
	\frac{u^n_i-u^{n-1}_i}{\Delta t}=& \frac{1}{\Delta x^2}\left(\alpha_{i+\frac{1}{2}}(u_{i+1}-u_i)-\alpha_{i-\frac{1}{2}}(u_{i}-u_{i-1})\right)\\
	u^{n-1}_i=&u^n_i- \frac{\Delta t}{\Delta x^2}\left(\alpha_{i+\frac{1}{2}}(u_{i+1}-u_i)-\alpha_{i-\frac{1}{2}}(u_{i}-u_{i-1})\right)
\end{align*}
Then we are able to put this into a linear equation as the section above. 
\begin{equation}
A =
\left(
\begin{array}{cccccccccc}
A_{0,0} & A_{0,1} & 0
&\cdots &
\cdots & \cdots & \cdots &
\cdots & 0 \\
A_{1,0} & A_{1,1} & A_{1,2} & \ddots &   & &  & &  \vdots \\
0 & A_{2,1} & A_{2,2} & A_{2,3} &
\ddots & &  &  & \vdots \\
\vdots & \ddots &  & \ddots & \ddots & 0 &  & & \vdots \\
\vdots &  & \ddots & \ddots & \ddots & \ddots & \ddots & & \vdots \\
\vdots & &  & 0 & A_{i,i-1} & A_{i,i} & A_{i,i+1} & \ddots & \vdots \\
\vdots & & &  & \ddots & \ddots & \ddots &\ddots  & 0 \\
\vdots & & & &  &\ddots  & \ddots &\ddots  & A_{N_x-1,N_x} \\
0 &\cdots & \cdots &\cdots & \cdots & \cdots  & 0 & A_{N_x,N_x-1} & A_{N_x,N_x}
\end{array}
\right)
\tag{14}
\end{equation}
Our matrix $A$ will then be tridiagonal with the values.
\begin{align} \label{eq:matrix_spatial}
A_{i,i} =& 1+\frac{\Delta t}{\Delta x^2}(\alpha(u_{i+\frac{1}{2}})+\alpha(u_{i-\frac{1}{2}}))\notag\\
A_{i,i-1} =&-\frac{\Delta t}{\Delta x^2}(\alpha(u_{i-\frac{1}{2}}))\notag\\
A_{i,i+1} =&-\frac{\Delta t}{\Delta x^2}(\alpha(u_{i+\frac{1}{2}}))
\end{align}                               
We can see that in the cases where we replace
$\alpha(u)$ with a function that contains $u$, we will get a nonlinear equation.

\paragraph{Boundary conditions.}
I have specified the boundary conditions in the first section to be zero. This has to be implemented in the linear function.
Our matrix A will then be
\begin{align*}
A_{0,0} &= 1,\\
A_{0,1} &= 0,\\
A_{N_x,N_x-1} &= 0,\\
A_{N_x,N_x} &= 1
\end{align*}
And the b array with known values will be
\begin{align*}
b_0 &= 0,\\
b_i &= u^{n-1}_i,\quad i=1,\ldots,N_x-1,\\
b_{N_x} &= 0 
\end{align*}
This decision affects the equilibrium state. 

\subsection{Linear equation}

The only solution for a linear equation is to replace $\alpha$ by a
constant.
\begin{equation}
	\alpha(u) = k
\end{equation}
We can then use our matrix from(\ref{eq:matrix_spatial}) and insert the function. This gives us
the matrix:
\begin{align*} \label{eq:matrix_constant}
A_{i,i} =& 1+\frac{\Delta t}{\Delta x^2}(2k)=1+\frac{2k\Delta t}{\Delta x^2}\\
A_{i,i-1} =&-\frac{\Delta t}{\Delta x^2}(k)=-\frac{k \Delta t}{\Delta x^2}\\
A_{i,i+1} =&-\frac{\Delta t}{\Delta x^2}(k)=-\frac{k \Delta t}{\Delta x^2}\\
\end{align*}                               
The movies under represents different values for $k$

% #MOVIE: [paramovies/constant.webm, height=300,width=375]constant

Here it is quite clearly that the size of $\alpha$ controls the speed of the
diffusion. A large $\alpha$ will get the function faster towards zero. 

\subsection{Nonlinear equation}

We will get a nonlinear solution for all $\alpha$ that include $u$. 
I will as in the previous section use Picard to solve it. There are also
other methods to solve nonlinear equation like Newton's method. This method often demands
a little bit more work, but is often more efficiency. 

\paragraph{Picard.}

\paragraph{Ordinary u.}
\begin{equation}
	\alpha(u) = u\\
\end{equation}

Since we only used CN for one step in the general calculation for $\alpha$, we need to use arithmetic mean
to be able to get values from $u$.
\begin{equation*}
	\alpha_{i\pm\frac{1}{2}} =\frac{1}{2}(\alpha_i+\alpha_{i\pm1}) 
\end{equation*}

Then we can insert $u$ in the matrix(\ref{eq:matrix_spatial}).
\begin{align*} 
A_{i,i} =& 1+\frac{\Delta t}{2\Delta x^2}(u_{i+1}+2u_{i}+u_{i-1})\\
A_{i,i-1} =&-\frac{\Delta t}{2\Delta x^2}(u_{i}+u_{i-1})\\
A_{i,i+1} =&-\frac{\Delta t}{2\Delta x^2}(u_{i+1}+u_{i})
\end{align*}                               
This gives us a nonlinear problem. We can here replace $u$ by $u\_$
as explained in the section \emph{Time Derivative}.
\begin{align*} 
A_{i,i} =& 1+\frac{\Delta t}{2\Delta x^2}(u\__{i+1}+2u\__{i}+u\__{i-1})\\
A_{i,i-1} =&-\frac{\Delta t}{2\Delta x^2}(u\__{i}+u\__{i-1})\\
A_{i,i+1} =&-\frac{\Delta t}{2\Delta x^2}(u\__{i+1}+u\__{i})
\end{align*}                               
I will do a couple of different setup for this function. First I will try to plot
a simulation where I only use the previous value $u\_$ in my matrix. Then I will add
an error measure that use several picard iterations.

% #MOVIE: [movies/spatial_ordinary.webm, height=300,width=375] ordinary
% #MOVIE: [movies/spatial_ordinary_picard_tol__2.webm,height=300,width=375]ordinary_picard with tol=10^-2
% #MOVIE: [movies/spatial_ordinary_picard_tol__5.webm, height=300,width=375]ordinary_picard with tol=10^-5

These results show us the behavior of $u\_$ when it varies from the previous
time step up to the current time step. The simulation without any picard
iteration, gives us a wall that spreads in each direction. This wall becomes
smaller the closer $u\_$ goes towards $u$. 

An important aspect to be aware of when working with
diffusion equation, is that the result after a certain time will be close to  equal in all
the simulations. This happens because of the equilibrium principle. An example
of this is the temperature in a room. The diffusion will cause it to go towards a smooth and
balanced temperature in each point of the room. Therefore it is more interesting
to look at the movement towards the equilibrium than the final result.

\subsection{Spatial derivation wrapped with an absolute value}

\begin{equation}
\alpha(u) = |\nabla(u)|
\end{equation}
Since we now have used CN one time on this, we can do it again. This will give
us integer numbers to use in $u$.
\begin{align*}
\alpha(u_{i+\frac{1}{2}}) =& \left|\nabla(u_{i+\frac{1}{2}})\right|\\
\alpha(u_{i+\frac{1}{2}}) =& \left|\frac{u_{i+1}-u_i}{\Delta x}\right|
\end{align*}
We can do the same thing for $\alpha_{i-\frac{1}{2}}$ and insert them both in
the matrix.
\begin{align*} 
A_{i,i} =& 1+\frac{\Delta t}{2\Delta x^2}(\left|\frac{u_i-u_{i-1}}{\Delta x}\right|+ \left|\frac{u_{i+1}-u_i}{\Delta x}\right|)\\
A_{i,i-1} =&-\frac{\Delta t}{2\Delta x^2}(\left|\frac{u_i-u_{i-1}}{\Delta x}\right|)\\
A_{i,i+1} =&-\frac{\Delta t}{2\Delta x^2}(\left|\frac{u_{i+1}-u_i}{\Delta x}\right|)
\end{align*}                               
To use this with picard, we need to replace $u\_=u$. We can also move $\Delta x$
out of the fraction, because $\Delta x > 0$.
\begin{align} \label{eq:matrix_absolute}
A_{i,i} =& 1+\frac{\Delta t}{2\Delta x^3}(\left|u\__i-u\__{i-1}\right|+ \left|u\__{i+1}-u\__i\right|)\notag\\
A_{i,i-1} =&-\frac{\Delta t}{2\Delta x^3}(\left|u\__i-u\__{i-1}\right|)\notag\\
A_{i,i+1} =&-\frac{\Delta t}{2\Delta x^3}(\left|u\__{i+1}-u\__i\right|)
\end{align}                               

% #MOVIE: [movies/spatial_abs_std.webm, height=300,width=375]abs_std
% #MOVIE: [movies/spatial_abs_std_picard_tol__5.webm,height=300,width=375]abs_std_picard tol = 10^-5

This also starts with a wall in each directions and then develops into smaller steps that nearly disappears when t gets larger.
While the picard solution gives a smoother curve. 

\subsection{Absolute value powered by m}

\begin{equation}
	\alpha(u) = |\nabla(u)|^m
\end{equation}
This equation is solved in the exact same as the equation above. We can then use
the matrix (\ref{eq:matrix_absolute}) and insert the power.
\begin{align*} 
A_{i,i} =& 1+\frac{\Delta t}{2\Delta x^3}(\left|u\__i-u\__{i-1}\right|^m+ \left|u\__{i+1}-u\__i\right|^m)\\
A_{i,i-1} =&-\frac{\Delta t}{2\Delta x^3}(\left|u\__i-u\__{i-1}\right|^m)\\
A_{i,i+1} =&-\frac{\Delta t}{2\Delta x^3}(\left|u\__{i+1}-u\__i\right|^m)
\end{align*}

% #MOVIE: [paramovies/abs_std_m.webm, height=300,width=375]Powered by m

Here the length of the simulation, $T$ is sat to be quite large. This is to show the effect of the power in
this function. We can see that it quite fast moves towards zero when the power is low. But it seems that the equilibrium state for all these are at zero. The power only affect the speed.


\section{Solution for the biological equation}

Now we can try to combine the reaction term with the spatial part. The reaction-diffusion equation(\ref{eq:reaction_diffusion}) can be inserted in a matrix. This is done by using the two linear equation from the earlier sections.
Since we only use the FK equation for $f(u)$, I will insert this in the matrix.
\begin{align} \label{eq:matrix_reaction_diffusion}
A_{i,i} =& 1+\Delta t(r(1-\frac{u^n_i}{m})+\frac{\alpha(u_{i+\frac{1}{2}})+\alpha(u_{i-\frac{1}{2}})}{\Delta x^2})\notag\\
A_{i,i-1} =&-\frac{\Delta t}{\Delta x^2}(\alpha(u_{i-\frac{1}{2}}))\notag\\
A_{i,i+1} =&-\frac{\Delta t}{\Delta x^2}(\alpha(u_{i+\frac{1}{2}}))
\end{align}                               

\paragraph{Constant.}
% #MOVIE: [paramovies/reac_constant.webm, height=300,width=375]r=1,M=1
% #MOVIE: [paramovies/reac_constant_r_0_5.webm, height=300,width=375]r=0.5,M=1
% #MOVIE: [paramovies/reac_constant_M_0_5.webm, height=300,width=375]r=1,M=0.5
% #MOVIE: [paramovies/reac_constant_M_0_5_r_0_5.webm, height=300,width=375]r=0.5,M=0.5
% #MOVIE: [paramovies/reac_constant_M_0_5_r_0_05.webm, height=300,width=375]r=0.05,M=0.5

\paragraph{Ordinary.}
% #MOVIE: [paramovies/reac_ordinary_r.webm, height=300,width=375]
% #MOVIE: [paramovies/reac_ordinary_M.webm, height=300,width=375]

\paragraph{Absolute value.}
% #MOVIE: [paramovies/reac_abs_std_r.webm, height=300,width=375]
% #MOVIE: [paramovies/reac_abs_std_M.webm, height=300,width=375]

\paragraph{Changes in the Boundary conditions.}

\paragraph{Dirichlet boundary condition.}
There are several ways to describe the boundary conditions for a differential equation.
In this paper I have used Dirichlet, where I decide that the boundaries shall be zero.
\begin{equation}
u = 0|_{\partial \Omega}
\end{equation}
This can be described with attach a rope to each side and then pull in the rope. It will always
be zero at the edges.

\subsection{Neumann boundary condition}

Another way to look at the boundary is by reflecting boundaries. It can reflect it by $u=0$ which change the sign at the boundary. The way that we often use this, is by Neumann boundary condition. This we can describe
\begin{equation}
 \frac{\partial u}{\partial n} \equiv \normalvec\cdot\nabla u = 0
\end{equation}
Where $\normalvec$ describes the normal vector.The boundaries can be describe  
\left.\frac{\partial}{\partial n}\right\vert_{x=0,L} =
\frac{\partial}{\partial x}
If we then discretize this at the boundaries, we get
\begin{align*}
\frac{u_{-1}^n - u_1^n}{2\Delta x} =& 0
u_{-1}^n =& u_1^n
\end{align*}
We can do the same thing for $L$.
Under I have made a simulation where I compare Dirichlet with Neumann. I run them
both on the spatial part to see how it affects the diffusion.


\begin{figure}[ht]
\begin{center}

\href{run:paramovies/reac_boundary.webm}{paramovies/reac_boundary.webm}

% alternative: \movie command that comes with beamer
% \movie[options]{paramovies/reac_boundary.webm}{paramovies/reac_boundary.webm}
\end{center}
\caption{}
\end{figure}


The interesting this here is to see that the Neumann preserves the area under the function, while Dirichlet smoothes it towards zero. The reason that Dirichlet goes towards zero is because of the value of the boundary. If we continue with the idea of a temperature in a room, the walls will have a large influence on the temperature. Neumann can be explained as wall covered with polystyrene(isopor),where the wall is totally isolated. With Dirichlet the walls has a specific temperature. In our example we sat it to zero, but any number would have caused an equilibrium state.  

\paragraph{Newton's method.}
Newtons method is another way to handle nonlinear equations. 
\begin{equation}
	F(u) = 0
\end{equation}
This method linearizes the equation by using Taylor series expansion around
$u\_$. This only keeping only the linear part.
\begin{align*}
	F(u) &= F(u_{-}) + F'(u_{-})(u - u_{-}) + \frac{1}{2}F''(u_{-})(u-u_{-})^2
	+\cdots  \\
	& \approx F(u_{-}) + F'(u_{-})(u - u_{-}) = \hat F(u)
\end{align*}
The linear equation will be
\begin{equation}
	u = u_{-} - \frac{F(u_{-})}{F'(u_{-})}
\end{equation}
This method require a Jacobian matrix. This means that we need to differentiate $F(u) = A(u)u - b(u)$ for all the 
values of $u$. Our $F$ will be
\begin{equation*}
F_i = A_{i,i-1}(u_{i-1},u_i)u_{i-1}+A_{i,i}(u_{i-1},u_{i},u_{i+1})u_{i}+A_{i,i+1}(u_i,u_{i+1})u_{i+1}-b_i(u^{n-1}_i)
\end{equation*}


\begin{align*}
J_{i,i}=& \frac{\partial F_i}{\partial u_i}=\frac{\partial A_{i,i-1}}{\partial u_i}u_{i-1}
+ \frac{\partial A_{i,i}}{\partial u_i}u_i+ A_{i,i}
+ \frac{\partial A_{i,i+1}}{\partial u_i}u_{i+1}
- \frac{\partial b_i}{\partial u_{i}}\\ 
J_{i,i-1}=& \frac{\partial F_i}{\partial u_{i-1}}=\frac{\partial A_{i,i-1}}{\partial u_{i-1}}u_{i-1}
+ A_{i,i-1}+\frac{\partial A_{i,i}}{\partial u_{i-1}}u_i
- \frac{\partial b_i}{\partial u_{i-1}}\\ 
J_{i,i+1}=& \frac{\partial F_i}{\partial u_{i+1}}= \frac{\partial A_{i,i}}{\partial u_{i+1}}u_i
+ \frac{\partial A_{i,i+1}}{\partial u_{i+1}}u_{i+1}+A_{i,i+1}
- \frac{\partial b_i}{\partial u_{i+1}}\\ 
\end{align*}
Since this demands a lot of calculating, I will show how to do it for the first
$J_{i,i}$. I calculate each subsection in one line
\begin{align*}
\frac{\partial A_{i,i-1}}{\partial u_i}u_{i-1}=&-\frac{\Delta t}{2\Delta x^2}(\alpha'(u_i)u_{i-1})\\
\frac{\partial A_{i,i}}{\partial u_i}u_i=&\frac{\Delta t}{2\Delta x^2}(2\alpha'(u_i)u_{i})\\
A_{i,i}=&1+\frac{\Delta t}{2\Delta x^2}(\alpha(u_{i+1})+2\alpha(u_i)+\alpha(u_{i-1}))\\
\frac{\partial A_{i,i+1}}{\partial u_i}u_{i+1}=&-\frac{\Delta t}{\Delta x^2}(\alpha'(u_i)u_{i+1})\\
- \frac{\partial b_i}{\partial u_{i}}=& -b'(u_i)\\
\end{align*}
This gives us the matrix
\begin{align*}
J_{i,i} =&-\frac{\Delta t}{2\Delta x^2}(\alpha'(u_i)u_{i-1})+\frac{\Delta t}{2\Delta x^2}(2\alpha'(u_i)u_{i})\\
+&1+\frac{\Delta t}{2\Delta x^2}(\alpha(u_{i+1})+2\alpha(u_i)+\alpha(u_{i-1}))-\frac{\Delta t}{2\Delta x^2}(\alpha'(u_i)u_{i+1})-b'(u_i)\\
J_{i,i-1} =&-\frac{\Delta t}{2\Delta x^2}(\alpha'(u_{i-1})u_{i-1})-\frac{\Delta t}{2\Delta x^2}(\alpha(u_{i-1})+\alpha(u_i))+\frac{\Delta t}{2\Delta x^2}(\alpha'(u_{i-1})u_{i})\\
J_{i,i+1} =&\frac{\Delta t}{2\Delta x^2}(\alpha'(u_{i+1})u_{i})-\frac{\Delta t}{2\Delta x^2}(\alpha'(u_{i+1})u_{i+1})-\frac{\Delta t}{2\Delta x^2}(\alpha(u_i)+\alpha(u_{i+1}))
\end{align*}
This matrix can be used to compute the Newton's method. We just need to replace $\alpha(u)$ by the function. 

\printindex

\end{document}