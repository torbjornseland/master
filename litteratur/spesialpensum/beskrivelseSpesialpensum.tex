\documentclass[norsk,11pt.a4paper]{article}

\usepackage[norsk]{babel}
\usepackage[utf8]{inputenc}
\usepackage{url}

\begin{document}

\title{Beskrivelse spesialpensum}
\author{for Torbjørn Seland}
\maketitle

\begin{itemize}
	\item Murray, James D. "Mathematical Biology: I. An Introduction." 2008
	\begin{itemize}
		\item Kap.10: \emph{Dynamics of Infectious Diseases(315-393)}\\
		Fokus på epidemier og modeller for dette. Beskrivelse av spredningsmodeller 
		for AIDS, HIV mfl.
		\item Kap.13: \emph{Biological waves: Single-Species Models(437-482)}\\
		Fokus på Fisher-Kolmogoroffs bølgelikning. Anvendelse og studier av likningen.
	\end{itemize}
	\item Murray, James D. "Mathematical Biology: II. Spatial Models and Biomedical Applications." 2004
	\begin{itemize}
		\item Kap.13: \emph{Geographic Spread and Control of Epidemics(661-720)}\\
		Fokus på spredning av epidemier. Modeller for rabiessmitte og SIR
	\end{itemize}
	\item Britton, Nicholas F. "Essential mathematical biology." Springer, 2003.
	\begin{itemize}
		\item Kap.3: \emph{Infectious Diseases(83-115)}\\
		Fokus på spredning epidemier. SIS og SIR
		\item Kap.5: \emph{Biological motion(147-172)}\\
		Fokus på bevegelsesmønstre.Diffusjonslikninger. Her vil "5.7: Spatial Spread of Epidemics" være
		sentralt
	\end{itemize}
	\item Brauer, Fred, and Carlos Castillo-Châavez. "Mathematical models in population biology and 		epidemiology." Springer, 2012.
	\begin{itemize}
		\item Kap.5: \emph{Continuous Models for Two Interacting Populations(171-226)}\\
		Fokus på interaksjon mellom arter. Hvordan påvirkes populasjonene et miljø med rovdyr og byttedyr.
		\item Kap.7: \emph{Basic Ideas of Mathematical Epidemiology(275-318)}\\
		Fokus modeller for epidemier. Se på faktorer som immunitet, demografisk plassering og populasjon.
	\end{itemize}
	\item Hjorth-Jensen, Morten. "Computational physics." Lecture notes . 2011, Part IV: Monte Carlo Methods
	\begin{itemize}
		\item Kap.11: \emph{Outline of the Monte Carlo Strategy(337-375)}\\
		Fokus på random walk. Se på hvordan dette kan anvendes for fysiske problemer.
	\end{itemize} 
\end{itemize}
\
\subsection*{Kommentar til pensumet}
Det totale sideantallet på dette spesialpensumet vil ligge på 375 sider, noe som vil tilsvare 75 sider per studiepoeng(5 studiepoeng).Om man sammenligner sideantallet med andre emner på samme nivå, burde kravene være oppfylt. Kurs som Inf5610(Matematisk modellering i medisin) og Inf4350(Grunnkurs i bioinformatikk) har følgende sideantall på 266\footnote{\url{http://www.uio.no/studier/emner/matnat/ifi/INF5610/h13/beskjeder/}}\footnote{\url{http://www.uio.no/studier/emner/matnat/ifi/INF5610/h13/overview.pdf}} og 164 sider\footnote{\url{http://www.uio.no/studier/emner/matnat/ifi/INF4350/v13/pensumliste/index.html}}. Dette tilsvarer 26,6 og 16,4 sider per studiepoeng.  
\end{document}
